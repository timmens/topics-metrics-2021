\usepackage[T1]{fontenc}
\usepackage[affil-it]{authblk}
\usepackage[document]{ragged2e}
\usepackage[flushleft]{threeparttable}
\usepackage[font=it]{caption}
\usepackage[left=1in, right=1in, top=1in, bottom=1in]{geometry}
\usepackage[noend]{algpseudocode}
\usepackage[plain]{algorithm}
\usepackage[scaled=0.85]{beramono}
\usepackage[scaled=0.90]{helvet}
\usepackage[tracking=true]{microtype}
\usepackage[utf8]{inputenc}
\usepackage{algorithmicx}
\usepackage{algorithm}
\usepackage{amsmath}
\usepackage{amssymb}
\usepackage{amsthm}
\usepackage{bbm}
\usepackage{bm}
\usepackage{bold-extra}
\usepackage{booktabs}
\usepackage{comment}
\usepackage{enumerate}
\usepackage{fancybox}
\usepackage{fontawesome5}
\usepackage{graphicx}
\usepackage{hyperref}
\usepackage{listings}
\usepackage{longtable}
\usepackage{lstbayes}
\usepackage{mathpazo}
\usepackage{mathtools}
\usepackage{multirow}
\usepackage{nicefrac}
\usepackage{scalerel}
\usepackage{setspace}
\usepackage{soul}
\usepackage{sourcesanspro}
\usepackage{subcaption}
\usepackage{tabularx}
\usepackage{textpos}
\usepackage{tgpagella}
\usepackage{tikz}
\usepackage{titlesec}
\usepackage{varwidth}
\usepackage{xcolor}
\usepackage{xfrac}
\usepackage{xpatch}

\counterwithin{figure}{section}
\usetikzlibrary{arrows}
\usetikzlibrary{positioning}

\usepackage[skins,theorems]{tcolorbox}
\tcbset{highlight math style={enhanced,colframe=red,colback=white,arc=0pt,boxrule=1pt}}

%%%%%%%%%%%%%%%%%%%%%%%%%%%%%%%%%%%%%%%%%%%%%%%%%%
% Bibliography settings
%%%%%%%%%%%%%%%%%%%%%%%%%%%%%%%%%%%%%%%%%%%%%%%%%%
\usepackage[backend=biber, style=apa, autocite=inline, doi=false, url=false]{biblatex}
\DeclareLanguageMapping{english}{english-apa}
\setcounter{biburlnumpenalty}{85}  % for urls that are too long
\addbibresource{bibliography.bib}
\renewcommand*{\bibfont}{\footnotesize}
\AtEveryBibitem{%
  \clearfield{volume}%
  \clearfield{number}%
  \clearfield{pages}%
}

%%%%%%%%%%%%%%%%%%%%%%%%%%%%%%%%%%%%%%%%%%%%%%%%%%
% Theorem environment etc.
%%%%%%%%%%%%%%%%%%%%%%%%%%%%%%%%%%%%%%%%%%%%%%%%%%
\theoremstyle{definition}
\newtheorem{definition}{Definition}
\newtheorem{assumption}{Assumption}
\newtheorem*{unnumdef}{Definition}

\theoremstyle{plain}
\newtheorem{proposition}{Proposition}
\newtheorem{theorem}{Theorem}
\newtheorem{lemma}{Lemma}
\newtheorem{corollary}{Corollary}

\theoremstyle{remark}
\newtheorem*{remark}{Remark}
\newtheorem*{note}{Note}
%%%%%%%%%%%%%%%%%%%%%%%%%%%%%%%%%%%%%%%%%%%%%%%%%%

%%%%%%%%%%%%%%%%%%%%%%%%%%%%%%%%%%%%%%%%%%%%%%%%%%
% Algorithm
%%%%%%%%%%%%%%%%%%%%%%%%%%%%%%%%%%%%%%%%%%%%%%%%%%
\makeatletter
\xpatchcmd{\algorithmic}{\itemsep\z@}{\itemsep=1ex plus1pt}{}{}
\makeatother

%%%%%%%%%%%%%%%%%%%%%%%%%%%%%%%%%%%%%%%%%%%%%%%%%%
% Math definitions
%%%%%%%%%%%%%%%%%%%%%%%%%%%%%%%%%%%%%%%%%%%%%%%%%%
\newcommand\indicator[1]{\scaleobj{1.4}{\mathbbm{1}} \! \left\{#1\right\}}
\newcommand\expectation[1]{\mathbb{E}\left[#1\right]}

\DeclareMathOperator*{\argmax}{argmax}

%%%%%%%%%%%%%%%%%%%%%%%%%%%%%%%%%%%%%%%%%%%%%%%%%%
% Colors
%%%%%%%%%%%%%%%%%%%%%%%%%%%%%%%%%%%%%%%%%%%%%%%%%%
\definecolor{bonnblue}{RGB}{4, 83, 156}  % hex: #04539C
\definecolor{bonngrey}{RGB}{148, 147, 132}  % hex: #8A9384
\definecolor{bonnyellow}{RGB}{251, 187, 6}  % hex: #FBBB06
\definecolor{red}{RGB}{65, 16, 16}  % hex: #411010

%%%%%%%%%%%%%%%%%%%%%%%%%%%%%%%%%%%%%%%%%%%%%%%%%%
% Auxiliary commands
%%%%%%%%%%%%%%%%%%%%%%%%%%%%%%%%%%%%%%%%%%%%%%%%%%

% \graphicspath{files/} % graphics path

\newcommand{\labelitem}{\tikz\draw[black,fill=bonnblue] (0,0) circle (.35ex); \,\,}

\newcommand{\blue}[1]{\textcolor{bonnblue}{#1}}
\newcommand{\yellow}[1]{\textcolor{bonnyellow}{#1}}
\newcommand{\grey}[1]{\textcolor{bonngrey}{#1}}
\newcommand{\red}[1]{\textcolor{red}{#1}}
