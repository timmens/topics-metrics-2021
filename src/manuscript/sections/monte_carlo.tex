\section{Monte-Carlo Study}
\label{section:monte_carlo}

In this last section I test the aforementioned extension using a simulation study.

The Monte-Carlo design compares an application of Algorithm \ref{algorithm:2} for $d \in
\{2, 4\}$. To gain a better understanding on the criticallity of Assumption
\ref{assumption:1}, and its relaxation, I consider a parametrized functional regressor
which allows me to choose the level of local variation.

\paragraph{Setup.}

The data generating process is setup as follows. For a given smoothness parameter $\nu
\in \{0.5, 1.5, 2.5\}$, the functional regressors $X_i$ are simulated as a mean-zero
Gaussian process with a Matern covariance kernel using length scale parameter $\ell =
0.1$ and smoothness parameter $\nu$ ---a mathematical description of the Matern kernel
and its relation to Assumption \ref{assumption:1} is provided in the next paragraph. The
process is observed for $T=100$ periods on an equidistant grid. I compare how the method
performs for $S \in \{0, 1, 2\}$ points-of-impact. In the case of $S = 1$ the location
is given by $\tau_1 = 49$. And in the case of $S = 2$ we have $(\tau_1, \tau_2) = (24,
49)$.  Given the number of points-of-impact $S$, the coefficient vectors are fixed
with: $\beta_0 = (1)$, $\beta_1 = (1, 2)$ and $\beta_2 = (1, 2, -1)$. The outcomes are
then simulated using
\[
    y_i = \beta_{S, 0} + \sum_{r = 1}^S \beta_{S, r} \, X_i(\tau_r) + \epsilon_i \,,
\]
where $\epsilon_i$ is an i.i.d. Gaussian error with $Var(\epsilon_i) = 1/2$. Note that
$\beta_{S, r}$ denotes the $r$-th entry of the $(S+1)$-dimensional coefficient vector,
in the case of $S$ number of points-of-impact. The number of observations is fixed to $n
= 100$. In principle, it would be interesting to see how the results depend on the
sample size; however, for the sake of clarity I refrain from analyzing this dimension.

Figure \ref{figure:process_scale_comparison} illustrates three simulated sample paths of
the functional regressor, for the three different smoothness parameters $\nu$. As is
clearly visible, for small $\nu$ (top row) the process posseses a lot of local
variation, while for larger $\nu$ (bottom row) the process is much smoother with a low
level of variation.


\begin{figure}

\centering
\begin{subfigure}[b]{\textwidth}
\begin{tcolorbox}[standard jigsaw, opacityback=0, top=0pt, left=0pt, right=0pt, bottom=0pt]
    \includegraphics[height=0.2\pdfpageheight,
    width=0.98\textwidth]{../../bld/figures/process_scale0.5}
\end{tcolorbox}
\end{subfigure}

\hfill

\begin{subfigure}[b]{\textwidth}
\centering
\begin{tcolorbox}[standard jigsaw, opacityback=0, top=0pt, left=0pt, right=0pt, bottom=0pt]
    \includegraphics[height=0.2\pdfpageheight,
    width=0.98\textwidth]{../../bld/figures/process_scale1.5}
\end{tcolorbox}
\end{subfigure}

\hfill

\begin{subfigure}[b]{\textwidth}
\centering
\begin{tcolorbox}[standard jigsaw, opacityback=0, top=0pt, left=0pt, right=0pt, bottom=0pt]
    \includegraphics[height=0.2\pdfpageheight,
    width=0.98\textwidth]{../../bld/figures/process_scale2.5}
\end{tcolorbox}
\end{subfigure}

\caption{Simulated trajectories of a Gaussian process with Matern kernel and differing
smoothness parameter; top: $\nu = 0.5$; center: $\nu = 1.5$; bottom: $\nu = 2.5$.}
\label{figure:process_scale_comparison}
\end{figure}


\paragraph{The Kernel.}

For this study I choose kernels from the Matern class. Explicitly, the kernel is defined
by

\[
    \sigma(s, t) = \frac{2^{1 - \nu}}{\Gamma(\nu)} \left(
    \frac{\sqrt{2 \nu} |s - t|}{\ell} \right)^\nu K_{\nu} \left( \frac{\sqrt{2 \nu} |s -
    t|}{\ell} \right) \,,
\]
with length scale parameter $\ell > 0$ and smoothness parameter $\nu > 0$. Note that
$\Gamma$ denotes the usual gamma function, while $K_{\nu}$ is a modified Bessel
function. For a more detailed reference on the components of the Matern kernel, as well
as a reference for the following properties, see \cite{Rasmussen2006}. The above
expression is hard to work with. Luckily, for the special cases $\nu \in \{0.5, 1.5,
2.5\}$ it simplifies dramatically, as is shown in Table \ref{table:matern_kernel}, where
I define $z = |s - t|$.

\smallskip

\begin{remark}
I must note that neither the Matern kernel with $\nu = 1.5$ nor with $\nu = 2.5$ satisfy
Assumption \ref{assumption:1}. This is because $\inf \{ c(t) : t \in [0, 1] \} = 0$ in
these cases. That means that the covariance at the diagonal does drop off fast enough.
Furthermore, an extension of the assumption to the case $\kappa < 4$ is not of help in
this case either, as the partial derivative $\partial \omega / \partial z$ is not
defined at $z = 0$ for any $\kappa > 1$. To analyze the extended method properly one
would need to use a kernel satisfying an extended version of Assumption
\ref{assumption:1} with $\kappa \in [2, 4)$. It proved difficult to find a reasonable
kernel for this case. As the sample paths induced by the Matern kernel with high $\nu$
are fairly standard, the results should, nevertheless, still tell us something of
relevance about the underlying method.
\end{remark}


\begin{table}
    \renewcommand{\arraystretch}{2}
    \centering
    \begin{tabular}{c|c}
        $\nu$ & $\sigma_{\text{Matern}}(z)$ \\ \hline
        $1/2$ & $\exp \left( - z / \ell \right)$\\
        $3/2$ & $\left(1 + \sqrt{3} z / \ell \right)\exp \left( - \sqrt{3}z / \ell \right)$\\
        $5/2$ & $\left(1 + \sqrt{5} z / \ell + 5z^2 / (3\ell^2) \right)\exp \left( -
        \sqrt{5}z / \ell \right)$
    \end{tabular}
    \caption{Simple expressions of the Matern kernel for special cases of the smoothness
    parameter $\nu$; see \cite{Rasmussen2006}. Here $z = |s - t|$.}
    \label{table:matern_kernel}
\end{table}



\paragraph{Monte-Carlo Design.}

I perform 500 Monte-Carlo repetitions over the parameter grid spanned by $d \in \{2, 4\},
S \in \{0, 1, 2\}$ and $\nu \in \{0.5, 1.5, 2.5\}$. The results are visualized using
frequency plots that summarize the estimated points-of-impact over \emph{all}
Monte-Carlo repetitions. A detailed explaination follows in the next paragraph.
Alternatively, one could have computed e.g. the Hausdorff-distance between the true and
estimated points-of-impact in each simulation run, or reported the average number of
estimated points-of-impact. For the sake of brevity I stick to one way of reporting the
results. Furthermore, in this study I focus only on the estimation of the
points-of-impact and not on the subsequent estimation of the coefficient parameters.


\paragraph{Results.}

Before we consider the actual results, let us think about what we may expect. The slope
parameter corresponding to the second point-of-impact is significantly smaller than the
one corresponding to the first point-of-impact. Hence, in the case of $S = 2$ we should
expect that the method finds the first point-of-impact at least as often. We also expect
that the precision of the method decreases in $\nu$, i.e. the smoother the functional
regressor the less precise the estimates. What we hope to see is that for the case $d =
4$, i.e. when using the fourth-order finite difference, the performance of the method
increases in the smoother $\nu = 2.5$ case.

Figure \ref{figure:monte_carlo_results_order2} summarizes the results when applying the
standard algorithm ($d = 2$). The top row shows the case of no points-of-impact ($S =
0$), the center row depicts the case of one point-of-impact ($S = 1$) and the bottom row
exhibits the case of two points-of-impact ($S = 2$). The results are consistent with our
expectations. The plot can be understood as follows: In the middle subfigure we see that
for the $\nu = 0.5$ case the true location ($49$) makes up more than $50\%$ of all the
estimated locations.

Even though Assumption \ref{assumption:1} is not satisfied for $\nu \in \{1.5, 2.5\}$,
the estimated locations form a cluster around the true points. This is in line with the
argumentation of \cite{Kneip2020}, in that the smooth functional regressors make it
harder for the method to distinguish between the influence of neighboring points.

Figure \ref{figure:monte_carlo_results_order4} depicts the case when employing the
fourth-order finite difference ($d = 4$). The image is very similar to the above figure.
However, a main difference is that the precision is lower than in the $d=2$ case.
Especially in the smooth cases the method performs worse, as there are no sharp
peaks and the estimated points are spread over a very large area.

The observation from Figure \ref{figure:monte_carlo_results_order4} may be due to
several reasons. First, as stated in the aformentioned remark, the kernel used to
simulate the functional regressors does not satisfy the extended Assumption
\ref{assumption:1}. The method remains to be tested with such a kernel. A
counter-argument would be that in the coarsely discretized case the regressors will
almost certainly look similar to the ones I am using here. Second, the fourth-order
finite difference formula may not be precise enough. One could think about modifying
this measure slightly to get a more precise criterion function.

\begin{figure}

\begin{subfigure}[b]{\textwidth}
\centering
\includegraphics[width=\textwidth]{../../bld/figures/monte_carlo/barplot_2_0}
\end{subfigure}

\begin{subfigure}[b]{\textwidth}
\centering
\includegraphics[width=\textwidth]{../../bld/figures/monte_carlo/barplot_2_1}
\end{subfigure}

\begin{subfigure}[b]{\textwidth}
\centering
\includegraphics[width=\textwidth]{../../bld/figures/monte_carlo/barplot_2_2}
\end{subfigure}

\caption{Results from 500 Monte-Carlo repetitions. Length scales are differentiated
using color; orange: $\nu = 0.5$; green: $\nu = 1.5$; blue: $\nu = 2.5$. In this
Monte-Carlo run Algorithm \ref{algorithm:2} was used with order $d = 2$. True
points-of-impact are depicted by the red vertical line.}
\label{figure:monte_carlo_results_order2}
\end{figure}

\begin{figure}

\begin{subfigure}[b]{\textwidth}
\centering
\includegraphics[width=\textwidth]{../../bld/figures/monte_carlo/barplot_4_0.pdf}
\end{subfigure}

\begin{subfigure}[b]{\textwidth}
\centering
\includegraphics[width=\textwidth]{../../bld/figures/monte_carlo/barplot_4_1.pdf}
\end{subfigure}

\begin{subfigure}[b]{\textwidth}
\centering
\includegraphics[width=\textwidth]{../../bld/figures/monte_carlo/barplot_4_2.pdf}
\end{subfigure}

\caption{Results from 500 Monte-Carlo repetitions. Length scales are differentiated
using color; orange: $\nu = 0.5$; green: $\nu = 1.5$; blue: $\nu = 2.5$. In this
Monte-Carlo run Algorithm \ref{algorithm:2} was used with order $d = 4$. True
points-of-impact are depicted by the red vertical line.}
\label{figure:monte_carlo_results_order4}
\end{figure}
