\section{Monte-Carlo Study}
\label{section:monte_carlo}

In this last section I test the aforementioned extension using a simulation study.

The Monte-Carlo design compares an application of Algorithm \ref{algorithm:2} with for
$d \in \{2, 4\}$. To gain a better understanding on the criticallity of Assumption
\ref{assumption:1}, and its relaxation, I consider a parametrized functional regressor
which allows me to choose the level of local variation. Moreover, I vary the number of
observed data points to analyze sample size effects.

\paragraph{Setup.}

The data generating process is setup as follows. For a given length scale $\ell \in
\{0.1, 0.05, 0.01\}$, the functional regressors $X_i$ are simulated as a mean-zero
Gaussian process with kernel $\sigma(s, t) = \exp\left(-|s - t|^2 / 2 \ell^2\right)$.
The process is observed for $T=100$ periods on an equidistant grid. I compare how the
method performs for $S \in \{0, 1, 2\}$ points-of-impact. In the case of $S = 1$ the
location is given by $\tau_1 = 49$. And in the case of $S = 2$ we have $(\tau_1, \tau_2)
= (24, 49)$. Given the number of points-of-impact $S$, the coefficient vectors are
fixed, with: $\beta_0 = (1)$, $\beta_1 = (1, 2)$ and $\beta_2 = (1, 2, -1)$. The
outcomes are then simulated using

\[
    y_i = \beta_{S, 0} + \sum_{r = 1}^S \beta_{S, r} \, X_i(\tau_r) + \epsilon_i \,,
\]
where $\epsilon_i$ is a Gaussian i.i.d. error with $Var(\epsilon_i) = 1/2$. Note that
$\beta_{S, r}$ denotes the $r$-th entry of the $(S+1)$-dimensional coefficient vector,
in the case of $S$ number of points-of-impact.

In Figure \ref{figure:process_scale_comparison} I illustrate three simulated sample
paths of the functional regressor, for the three different length scales $\ell$. As is
clearly visible, for a small length scale the process posseses a lot of local variation,
while for a larger length scale the process is very smooth with a low level of
variation.

I perform 100 Monte-Carlo repetitions over the parameter grid spanned by $d = \{2, 4\},
S = \{0, 1, 2\}, \ell = \{0.1, 0.05, 0.01\}$ and $n = \{20, 50, 100\}$. I choose to
illustrate the results using frequency plots that capture all estimated points-of-impact
over the whole repetition. Alternatively, one could have computed, for example, the
Hausdorff-distance between the true and estimated points-of-impact in each simulation
run, or reported the average number of estimated points-of-impact. For the sake of
brevity I stick to only one way of reporting the results. Furthermore, in this study I
focus only on the estimation of the points-of-impact and not on the subsequent
estimation of the coefficient parameters.


\begin{figure}

\centering
\begin{subfigure}[b]{\textwidth}
\begin{tcolorbox}[standard jigsaw, opacityback=0, top=0pt, left=0pt, right=0pt, bottom=0pt]
    \includegraphics[height=0.2\pdfpageheight, width=0.98\textwidth]{../../bld/figures/process_scale0.01}
\end{tcolorbox}
\end{subfigure}

\hfill

\begin{subfigure}[b]{\textwidth}
\centering
\begin{tcolorbox}[standard jigsaw, opacityback=0, top=0pt, left=0pt, right=0pt, bottom=0pt]
    \includegraphics[height=0.2\pdfpageheight, width=0.98\textwidth]{../../bld/figures/process_scale0.05}
\end{tcolorbox}
\end{subfigure}

\hfill

\begin{subfigure}[b]{\textwidth}
\centering
\begin{tcolorbox}[standard jigsaw, opacityback=0, top=0pt, left=0pt, right=0pt, bottom=0pt]
    \includegraphics[height=0.2\pdfpageheight, width=0.98\textwidth]{../../bld/figures/process_scale0.1}
\end{tcolorbox}
\end{subfigure}

\caption{Simulated trajectories of a Gaussian process with radial basis function kernel
and differing length scale parameter; top: $\ell = 0.01$; center: $\ell = 0.05$; bottom:
$\ell = 0.1$.}
\label{figure:process_scale_comparison}
\end{figure}


\paragraph{Results.}
