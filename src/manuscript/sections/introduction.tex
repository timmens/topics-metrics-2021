\section{Introduction}

In this essay I build on the paper \emph{"Superconsistent estimation of points of impact
in non-parametric regression with functional predictors"} by \cite{Kneip2020}. The paper
considers the classical problem of scalar outcome prediction in the case of functional
regressors. The main variation is the structural assumption that there are certain time
points for which the outcome depends \emph{only} on the functional regressors observed
at these time points. These points are called \emph{points-of-impact}. The number and
location of the points-of-impact is assumed to be unknown and must be estimated from the
data directly. A similar problem has been examined in \cite{Kneip2016}, where the model
presumes a linear influence of the functional regressors at the points-of-impact plus a
common effect of the whole trajectory. In \cite{Kneip2020} the effect of the whole
trajectory is ignored; however, the relationship between the functional regressors at
the points-of-impact and the outcome are modeled in a non-parametric framework.
Unexpectedly, identification of the points-of-impact does not require harsh assumptions
on the link between regressors and outcome, but rather on the structure of the
functional regressors themselves. We will see that a useful abstraction level is to
require assumptions on the smoothness of the covariance kernel of the functional
regressor on and off the diagonal. These assumptions naturally lead to a criterion
function which can then be used to estimate the points-of-impact. Following a remark in
\cite{Kneip2020}, I consider a closely related assumption on the covariance kernel
which, in principle, allows for a greater number of regressor processes to be modeled.
This new assumption then implies a different criterion function. To see how these
different criteria compare I run Monte-Carlo studies. The \textsf{R}-code that
implements the new features is hosted on GitHub
(\url{https://github.com/timmens/fdapoi}) and builds on the original package
corresponding to the paper; see \url{https://github.com/lidom/fdapoi}.

The rest of this essay is structured as follows: In Section \ref{section:review} I
present the mathematical model and review the main assumptions and theorems from the
paper. Section \ref{section:extension} considers the aforementioned extension from a
theoretical and computational viewpoint. And at last, in Section
\ref{section:monte_carlo} I present the Monte-Carlo study.
