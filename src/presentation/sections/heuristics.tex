\section{Heuristics}


\begin{frame}{Plan}

    \vspace{-0.5cm}
    \begin{table}[]
    \renewcommand{\arraystretch}{1.5}
        \begin{tabular}{ll}
          \labelitem Simple linear model & $Y_i = \beta_0 + \sum_{s = 1}^S \beta_s X_i(\tau_s) + \epsilon_i$\\
          \labelitem Gaussian Process Regressors &  $X_i \sim \mathcal{GP}(0, \sigma)$\\
        \end{tabular}
    \end{table}

    \vspace{0.5cm}
    \begin{table}[]
    \renewcommand{\arraystretch}{1.5}
        \begin{tabular}{ll}
            \blue{Task:} & Think of how to identify/estimate $\tau_s$ for different $\sigma$\\
            \grey{First insight:} & Investigate object $\expectation{Y_i X_i(t)}$\\
            \yellow{Kernels:} & White, Brownian Motion, Matern, RBF
        \end{tabular}
    \end{table}

\end{frame}


\begin{frame}{General Representation}{Equation 1}

    \vspace{-0.5cm}
    \begin{align*}
    \label{eq:1}
        f(t) & \stackrel{def}{=} \expectation{Y_i X_i(t)}\\
             &= \expectation{\left(\beta_0 + \sum_{s = 1}^S \beta_s X_i(\tau_s) + \epsilon_i\right) X_i(t)}\\
             &= \sum_{s = 1}^S \beta_s \, \expectation{X_i(\tau_s) X_i(t)}\\
             &= \sum_{s = 1}^S \beta_s \, \sigma(t, \tau_s)
    \end{align*}

\end{frame}


\begin{frame}{White Kernel}

\vspace{-1.2cm}
\begin{centering}
\begin{tikzpicture}

    \draw node at (8.35, 4) {$\sigma(s, t) = \indicator{s = t}$};
    \draw node at (9, 3) {$f(t) \stackrel{eq.1}{=} \sum_{s = 1}^S \beta_s \, \sigma(t, \tau_s)$};
    \draw node at (9.65, 2.2) {$= \sum_{s = 1}^S \beta_s \, \indicator{t = \tau_s}$};

    \draw[line width=0.25mm, ->] (0, -2.5) -- (0, 3.5) node [left] {$f(t)$};
    \draw[line width=0.25mm, ->] (-0.5, 0) -- (11, 0) node [below] {$t$};

    \draw (10, -0.2) -- (10, 0.2) node [below=0.4cm] {$1$};
    \draw node at (-0.2, -0.2) {$0$};

    \draw[line width=0.25mm] (2.5, -0.2) -- (2.5, 0) node [below=0.2cm] {$\tau_1$};
    \draw[line width=0.25mm] (5, -0.2) -- (5, 0) node [below=0.2cm] {$\tau_2$};
    \draw[line width=0.25mm] (7.5, 0) -- (7.5, 0.2) node [above] {$\tau_3$};

    \draw[line width=0.25mm, color=bonnblue] (0.02, 0.03) -- (9.98, 0.03) {};

    \draw[line width=0.25mm, color=bonnblue] (2.5, 0.03) -- (2.5, 1) node [left] {$\beta_1$};
    \draw[line width=0.25mm, color=bonnblue] (5, 0) -- (5, 3) node [left] {$\beta_2$};
    \draw[line width=0.25mm, color=bonnblue] (7.5, -0.03) -- (7.5, -2) node [left] {$\beta_3$};

    \draw[line width=0.2mm, color=bonngrey, dashed] (1.9, 1) -- (0, 1) node [left] {$1$};
    \draw[line width=0.2mm, color=bonngrey, dashed] (4.4, 3) -- (0, 3) node [left] {$3$};
    \draw[line width=0.2mm, color=bonngrey, dashed] (6.9, -2) -- (0, -2) node [left] {$-2$};

\end{tikzpicture}
\end{centering}
    
\end{frame}


\begin{frame}{Brownian Motion}

\vspace{-1.2cm}
\begin{centering}
\begin{tikzpicture}

    \draw node at (8.35, 4) {$\sigma(s, t) = 2 \min\{s, t\}$};
    \draw node at (9, 3) {$f(t) \stackrel{eq.1}{=} \sum_{s = 1}^S \beta_s \, \sigma(t, \tau_s)$};
    \draw node at (9.65, 2.2) {$= \sum_{s = 1}^S \beta_s \, \min\{t, \tau_s\}$};

    \draw[line width=0.25mm, ->] (0, -2.5) -- (0, 3.5) node [left] {$f(t)$};
    \draw[line width=0.25mm, ->] (-0.5, 0) -- (11, 0) node [below] {$t$};

    \draw (10, -0.2) -- (10, 0.2) node [below=0.4cm] {$1$};
    \draw node at (-0.2, -0.2) {$0$};

    \draw[line width=0.25mm] (2.5, -0.2) -- (2.5, 0) node [below=0.2cm] {$\tau_1$};
    \draw[line width=0.25mm] (5, -0.2) -- (5, 0) node [below=0.2cm] {$\tau_2$};
    \draw[line width=0.25mm] (7.5, -0.2) -- (7.5, 0) node [below=0.2cm] {$\tau_3$};

    \draw[line width=0.25mm, dashed, color=bonngrey] (2.5, 0) -- (2.5, 1);
    \draw[line width=0.25mm, dashed, color=bonngrey] (5, 0) -- (5, 1.5);
    \draw[line width=0.25mm, dashed, color=bonngrey] (7.5,0) -- (7.5, 0.5);

    \draw[line width=0.25mm, color=bonnblue] (0.02, 0.03) -- (2.5, 1);
    \draw[line width=0.25mm, color=bonnblue] (2.5, 1) -- (5, 1.5);
    \draw[line width=0.25mm, color=bonnblue] (5, 1.5) -- (7.5, 0.5);
    \draw[line width=0.25mm, color=bonnblue] (7.5, 0.5) -- (10, 0.5);

\end{tikzpicture}
\end{centering}
    
\end{frame}


\begin{frame}

    \begin{center}
        \includegraphics[width=0.95\textwidth]{../../bld/figures/cross-covariance.png}
    \end{center}


\end{frame}


\begin{frame}{Finite Difference and Differentiability}
    \vspace{-1cm}
    Let $h: [0, 1] \to \mathbb{R}$.

    \vspace{0.25cm}
    \begin{paragraph} \blue{Definition.}\\
        \quad $h$ is differentiable at $t \in (0, 1)$ if
        $\lim_{\delta \to 0} (h(t + \delta) - h(t - \delta)) / \delta$ exists.
    \end{paragraph}

    \vspace{0.5cm}
    \begin{paragraph} \blue{Definition.}\\
        \quad $c(t, \delta) = h(t + \delta) - h(t - \delta)$ is the (first-order)
        centered difference of $h$.\\
        \quad For small $\delta$, $h'(t)$ is approximated by $c(t, \delta) / \delta$.
    \end{paragraph}

    \vspace{0.5cm}
    \begin{paragraph} \grey{Remark.}\\
        \quad Applying the first-order centered difference twice yields the second-order\\
        \quad difference $c^2(t, \delta) = h(t + \delta) + h(t + \delta) - 2h(t)$. For small
        $\delta$, $h''(t)$ is\\ \quad approximated by $c^2(t, \delta) / \delta$.
    \end{paragraph}

\end{frame}
